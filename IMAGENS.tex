\imagemgrande{Médicos e enfermeiros voluntários trabalhavam em ritmo intenso na noite de 9 de outubro de 2015 para atender centenas de refugiados que passavam por Idomeni rumo ao norte da Europa.}{./img/CSC_0047.jpg}\clearpage
\imagemgrande{Naquela noite, centenas de pacientes foram atendidos.}{./img/DSC_0033.jpg}\clearpage
\imagemgrande{Quando milhares de pessoas ficaram presas em Idomeni em novembro de 2016, o ACNUR e o MSF instalaram novas tendas para abrigar mais refugiados.}{./img/DSC_0055.jpg}\clearpage
\imagemgrande{Uma nova parte do centro de transição foi criada e chamada de “Campo B”, visto nesta imagem.}{./img/DSC_0057.jpg}\clearpage
\imagemgrande{As tendas com aquecimento do MSF na seção principal do centro de transição não tinham como abrigar todos os refugiados que chegavam ao local naquela época.}{./img/DSC_0071.jpg}\clearpage
\imagemgrande{Mesmo com a fronteira da República da Macedônia fechada para aqueles que não fossem nacionais de Afeganistão, Iraque e Síria, os ônibus com refugiados continuavam a chegar a Idomeni em grandes quantidades.}{./img/DSC_0075.jpg}\clearpage
\imagemgrande{Sem abrigo nas tendas “oficiais”, os refugiados montavam suas barracas ou dormiam sob colchões.}{./img/DSC_0114.jpg}\clearpage
\imagemgrande{Centenas de barracas se acumulavam próximas à fronteira com a República da Macedônia.}{./img/DSC_0119.jpg}\clearpage
\imagemgrande{Como esta imagem retrata, a fumaça negra das fogueiras se espalhava pelo centro de transição.}{./img/DSC_0120.jpg}\clearpage
\imagemgrande{As barracas bloqueavam os trilhos e impediam que os trens pudessem passar por Idomeni.}{./img/DSC_0125.jpg}\clearpage
\imagemgrande{Como havia pouco espaço no centro de transição, os refugiados foram forçados a erguer suas tendas em áreas mais afastadas.}{./img/DSC_0174.jpg}\clearpage
\imagemgrande{Em novembro de 2015, para enfrentar as baixas temperaturas do norte grego, os refugiados acendiam fogueiras. Alguns conseguiram cortar árvores para usar como lenha.}{./img/DSC_0052.jpg}\clearpage
\imagemgrande{Mas a maioria queimava o que conseguisse encontrar, como roupas, plástico, sapatos. A fumaça tomava conta do centro de transição.}{./img/DSC_0110.jpg}\clearpage
\imagemgrande{Refugiados se aquecem em fogueira.}{./img/DSC_0116.jpg}\clearpage
\imagemgrande{Refugiados acendem fogueira enquanto oficiais da República da Macedônia observam ao fundo.}{./img/DSC_0160.jpg}\clearpage
\imagemgrande{A impossibilidade de milhares de pessoas de seguir viagem aumentou a tensão no centro. O governo grego intensificou o policiamento para proteger os atores humanitários operando no local, os moradores e evitar conflitos entre refugiados.}{./img/DSC_0078.jpg}\clearpage
\imagemgrande{Como milhares de pessoas ficaram presas em Idomeni, as filas para receber alimentos e água eram sempre longas.}{./img/DSC_0062.jpg}\clearpage
\imagemgrande{Grupo de refugiados no centro de transição em Idomeni.}{./img/DSC_0083.jpg}\clearpage
\imagemgrande{O lixo acumulava-se nas caçambas e no chão rapidamente.}{./img/DSC_0098.jpg}\clearpage
\imagemgrande{A República da Macedônia reforçou o policiamento na frontera com a Grécia para evitar que nacionais de países não autorizados entrassem no país.}{./img/DSC_0123.jpg}\clearpage
\imagemgrande{Inicialmente, baixas cercas de arame farpado foram instaladas na divisa entre os dois países, assim como abrigos para os policiais macedonianos. Os refugiados se aglomeravam próximos a cerca, na expectativa de poder cruzar.}{./img/DSC_0127.jpg}\clearpage
\imagemgrande{Pedaços de metal e papel emaranhados nas árvores e nas cercas de arame farpado.}{./img/DSC_0135.jpg}\clearpage
\imagemgrande{As árvores secas do outono europeu.}{./img/DSC_0139.jpg}\clearpage
\imagemgrande{Homem protesta contra o fechamento da fronteira e pede para que “abram o caminho”.}{./img/DSC_0142.jpg}\clearpage
\imagemgrande{A poucos centímetros de distância, policiais macedonianos armados com equipamentos de choque garantiam que ninguém entraria no país pela linha do trem.}{./img/DSC_0147.jpg}\clearpage
\imagemgrande{Um pequeno check point foi criado pela República da Macedônia para autorizar a entrada de refugiados no país. No início de dezembro, uma cerca de mais de dois metros de altura, além de arames farpados, separavam Idomeni da ex-república iugoslava.}{./img/DSC_0197.jpg}\clearpage
\imagemgrande{Roupas e outros objetos agarrados no arame farpado em frente à cerca.}{./img/DSC_0218.jpg}\clearpage
\imagemgrande{Aqueles que ainda podiam cruzar a fronteira, acumulavam-se em longas filas no lado grego até que tivessem sua entrada liberada na República da Macedônia.}{./img/DSC_0184.jpg}\clearpage
\imagemgrande{Centenas de roupas e cobertores eram abandonadas diariamente pelos refugiados em Idomeni.}{./img/DSC_0198.jpg}\clearpage
\imagemgrande{As filas para cruzar a fronteira eram enormes.}{./img/DSC_0205.jpg}\clearpage
\imagemgrande{A polícia grega tentava organizar os refugiados dividindo-os em grupos menores para que a travessia não fosse tumultuosa.}{./img/DSC_0214.jpg}\clearpage
\imagemgrande{Havia muito lixo espalhado pelo centro de transição.}{./img/DSC_0220.jpg}\clearpage
\imagemgrande{A cerca impedia que refugiados entrassem no território macedoniano sem passar antes pelas autoridades.}{./img/DSC_0221.jpg}\clearpage
\imagemgrande{Voluntários de diversos países ocuparam um vagão de trem onde instalaram uma cozinha para preparar refeições para refugiados em dezembro de 2015.}{./img/DSC_0224.jpg}\clearpage
\imagemgrande{Em maio de 2016, milhares de refugiados voltaram a ficar sitiados em Idomeni após o fechamento total da rota dos Bálcãs.
}{./img/DSC_0494.jpg}\clearpage
\imagemgrande{As condições climáticas adversas, tornaram a situação dos refugiados ainda mais difícil uma vez que suas barracas ficavam em campos alagados e as condições sanitárias do centro de transição eram péssimas.}{./img/DSC_0497.jpg}\clearpage
\imagemgrande{Uma mulher lavava algo do lado de fora de sua tenda.}{./img/DSC_0499.jpg}\clearpage
\imagemgrande{Nesta época do ano, a chuva era intensa e constante.}{./img/DSC_0502.jpg}\clearpage
\imagemgrande{As plantações, onde milhares de refugiados acampavam, ficaram alagadas na maior parte do tempo.}{./img/DSC_0506.jpg}\clearpage
\imagemgrande{Isso provocava problemas de saúde em centenas de pessoas, segundo organizações médicas do local.}{./img/DSC_0509.jpg}\clearpage
\imagemgrande{Sem varais para colocar suas roupas, os refugiados precisavam improvisar.}{./img/DSC_0510.jpg}\clearpage
\imagemgrande{Ao fundo, estão algumas tendas brancas instaladas pelo MSF para tentar lidar com o fluxo extra de refugiados a partir de março de 2016.}{./img/DSC_0512.jpg}\clearpage
% \imagemgrande{Milhares de refugiados instalaram suas barracas ao redor daquela área.}{./img/DSC_0514.jpg}\clearpage
\imagemgrande{}{./img/DSC_0515.jpg}\clearpage
\imagemgrande{O MSF tentou ampliar o centro de transição mais uma vez para abrigar os novos refugiados, mas não conseguiu lidar com o alto número de pessoas sitiadas no local.}{./img/DSC_0516.jpg}\clearpage
\imagemgrande{Sem opção, grande parte dos refugiados tinha que acampar em terrenos alagados e lamacentos.}{./img/DSC_0517.jpg}\clearpage
% \imagemgrande{Alguns conseguiam espaço na linha do trem, onde havia menos água.}{./img/DSC_0520.jpg}\clearpage
\imagemgrande{Mas a maioria não tinha a mesma sorte.}{./img/DSC_0522.jpg}\clearpage
\imagemgrande{A rua que ligava o centro de transição ao “Campo B” era estreita, mas alguns refugiados conseguiram instalar suas tendas ali.}{./img/DSC_0528.jpg}\clearpage
\imagemgrande{As condições sanitárias do local eram tão ruins que lembravam áreas pobres em países em desenvolvimento.}{./img/DSC_0533.jpg}\clearpage
\imagemgrande{Algumas barracas menores foram instaladas pelo MSF.}{./img/DSC_0535.jpg}\clearpage
\imagemgrande{A vista da entrada do “Campo B”.}{./img/DSC_0536.jpg}\clearpage
\imagemgrande{}{./img/DSC_0541.jpg}\clearpage
\imagemgrande{Em outubro de 2016, cinco meses após a remoção do refugiados, Idomeni ainda guardava os vestígios da crise humanitária enfrentada pelo pequeno vilarejo. Milhares de roupas e outros objetos que pertenciam a refugiados foram abandonados no local onde antes existia o centro de transição.}{./img/DSC_0014.jpg}\clearpage